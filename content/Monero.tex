/chapter{Monero the Cryptocurrency}
/label{chapter:Monero-cryptocurrency}

final chapter ties it all together (? technical enough to skip rest of report? might make it too long - aim to redirect to other parts of report)


-curve ed25519
-address creation
    -basic
    -subaddresses
    -multisig
    -payment ID (integrated addresses)
-receipt of old transactions
    -scan blockchain, look at output addresses, use view key and transaction public key to see if any spend keys match
    -organize results
-building transaction
    -structure of a transaction (how the data is serialized/organized and transmitted)
    -obtain addresses (or subaddresses) of intended recipients, specify amounts intended for each
    -from sending amounts + fee, select a set of owned outputs with amounts sum(a)>=(sum(b)+fee), and set change = sum(a) - (sum(b)+fee)
    -construct outputs:
        -create transaction public key (or keys if at least one subaddress and >1 output)
        -create one-time output address for each output
        -commit to each output amount C(b)
        -create D-H 'amount' and 'mask' terms for each output commitment
        -range proof: borromean ring signature on each output amount
        -if payment ids, encrypt them
    -construct inputs:
        -create pseudo output commitments if transaction type rcttypesimple
        -select ring members for each input mlsag
        -calculate key images for each input
        -build MLSAG signatures on each input, signing a message that contains all other transaction info
    -fill out transaction data structure appropriately (continuous throughout transaction procedure)
-submission of transaction
    -verified, placed in mempool
-mining into blockchain
    -transactions organized into merkle tree, hashed
    -nonce searched until difficulty reached
    -block published to network
    -block accepted or rejected (orphaned); consensus mechanism
    -[potential] signature data pruned