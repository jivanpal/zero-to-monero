\chapter{Monero Addresses}
\label{chapter:addresses}

The ownership of digital currency stored in a blockchain is controlled by so-called `addresses'. Addresses are sent money which only the address-holder can spend.\footnote{Except with negligible probability.}

More specifically, an address owns the `outputs' from some transactions, which are like notes giving the address-holder spending rights to an `amount' of money. Such a note might say ``Address C now owns 5.3 XMR".

To spend an owned output, the address-holder references it as the input to a new transaction. This new transaction has outputs owned by other addresses (or by the sender's address, if the sender wants). A transaction's total input amount equals its total output amount, and once spent an owned output can't be respent. Carol, who has Address C, could reference that old output in a new transaction (e.g. ``In this transaction I'd like to spend that old output.") and add a note saying ``Address B now owns 5.3 XMR".

An address's balance is the sum of amounts contained in its unspent outputs.\footnote{Computer applications known as `wallets' are used to find and organize the outputs owned by an address, to maintain custody of its private keys for authoring new transactions, and to submit those transactions to the network for verification and inclusion in the blockchain.}

We discuss hiding the amount from observers in Chapter \ref{chapter:pedersen-commitments}, the structure of transactions in Chapter \ref{chapter:transactions} (which includes how to prove you are spending an owned and previously unspent output, without even revealing which output is being spent), and the money creation process and role of observers in Chapter \ref{chapter:blockchain}.



\section{User keys}
\label{sec:user-keys}

Unlike Bitcoin, Monero users have two sets of private/public keys, \((k^v, K^v)\) and \( (k^s, K^s) \), generated as described in Section \ref{ec:keys}.

The {\em address}\marginnote{src/ common/ base58.cpp {\tt encode\_ addr()}} of a user is the pair of public keys \((K^v, K^s)\). Her private keys will be the corresponding pair \( (k^v, k^s) \).\footnote{To communicate an address to other users, it is extremely common to encode it in base-58, a binary-to-text encoding scheme first created for Bitcoin \cite{base-58-encoding}. See \cite{luigi-address} for more details.}
\\

Using two sets of keys allows function segregation. The rationale will become clear later in this chapter, but for the moment let us call private key $k^v$ the {\em view key}, and $k^s$ the {\em spend key}. A person can use their view key to determine if their address owns an output, and their spend key will allow them to spend that output in a transaction (and retroactively figure out it has been spent).\footnote{It\marginnote{src/wallet/ api/wallet.cpp {\tt create()} wallet2.cpp {\tt generate()} {\tt get\_seed()}} is currently most common for the view key $k^v$ to equal $\mathcal{H}_n(k^s)$. This means a person only needs to save their spend key $k^s$ in order to access (view and spend) all of the outputs they own (spent and unspent). The spend key is typically represented as a series of 25 words (where the 25th word is a checksum). Other, less popular methods include: generating $k^v$ and $k^s$ as separate random numbers, or generating a random 12-word mnemonic $a$, where $k^s = {\tt sc\textunderscore reduce32}(\mathit{Keccak}(a))$ and $k^v = {\tt sc\textunderscore reduce32}(\mathit{Keccak}(\mathit{Keccak}(a)))$. \cite{luigi-address}}



\section{One-time addresses}
\label{sec:one-time-addresses}

To receive money, a Monero user may distribute their address to other users, who can then send it money via transaction outputs.

The address is never used directly.\footnote{The method described here is not enforced by the protocol, just by wallet implementation standards. This means an alternate wallet could follow the style of Bitcoin where recipients' addresses are included directly in transaction data. Such a non-compliant wallet would produce transaction outputs unusable by other wallets, and each Bitcoin-esque address could only be used once due to key image uniqueness.} Instead, a Diffie-Hellman-like exchange is applied to it, creating a unique {\em one-time address} for each transaction output to be paid to the user. In this way, even external observers who know all users’ addresses cannot use them to identify which user owns any given transaction output.\footnote{Except with negligible probability.}

%A recipient can spend their received outputs by signing a message with the one-time addresses, thereby proving they know the private keys and therefore own what they are spending. We will gradually flesh out this concept.
%\\

Let’s start with a very simple mock transaction, containing exactly one output --- a payment of `0' amount from Alice to Bob.

Bob has private/public keys $(k_B^v, k_B^s)$ and $(K_B^v, K_B^s)$, and Alice knows his public keys (his address). The mock transaction could proceed as follows \cite{cryptoNoteWhitePaper}:

\begin{enumerate}
	\item Alice generates a random number $r \in_R \mathbb{Z}_l$, and calculates the one-time address\footnote{In\marginnote{src/crypto/ crypto.cpp {\tt generate\_ key\_deri- vation()}} Monero, every instance (including when it's used for other parts of the transaction) of $r k^v G$ is multiplied by the cofactor 8, so in this case Alice computes $8*r K^v_B$ and Bob computes $8*k^v_B r G$. As far as we can tell this serves no purpose (but it {\em is} a rule that users must follow). Multiplying by the cofactor ensures the resulting point is in $G$'s subgroup, but if $R$ and $K^v$ don't share a subgroup to begin with, then the discrete logs $r$ and $k^v$ can't be used to make a shared secret regardless.}\vspace{.175cm}
	\[K^o  = \mathcal{H}_n(r K_B^v)G + K_B^s\marginnote{src/crypto/ crypto.cpp {\tt derive\_pu- blic\_key()}}\]

	\item Alice sets $K^o$ as the addressee of the payment, adds the output amount `0' and the value $r G$ to the transaction data, and submits it to the network.

	\item Bob\marginnote{src/crypto/ crypto.cpp {\tt derive\_ subaddress\_ public\_ key()}} receives the data and sees the values $r G$ and $K^o$. He can calculate $k_B^v r G = r K_B^v$. He can then calculate $K'^s_B = K^o - \mathcal{H}_n(r K_B^v)G$. When he sees that $K'^s_B = K_B^s$, he knows the output is addressed to him.\footnote{$K'^s_B $ is computed with {\tt derive\_subaddress\_public\_key()} because normal address spend keys are stored at index 0 in the spend key lookup table, while subaddresses are at indices 1,2... This will make sense soon: see Section \ref{sec:subaddresses}.}

	The private key $k_B^v$ is called the `view key' because anyone who has it (and Bob’s public spend key $K_B^s$) can calculate $K'^s_B$ for every transaction output in the blockchain (record of transactions), and ‘view’ which ones belong to Bob.

	\item The one-time keys for the output are:\vspace{.175cm}
	\begin{align*}
		K^o &= \mathcal{H}_n(r K_B^v)G + K_B^s = (\mathcal{H}_n(r K_B^v) + k_B^s)G  \\ 
		k^o &= \mathcal{H}_n(r K_B^v) + k_B^s
	\end{align*}
\end{enumerate}

To spend his `0' amount output [sic] in a new transaction, all Bob needs to do is prove ownership by signing a message with the one-time key $K^o$. The private key $k_B^s$ is the `spend key' since it is required for proving output ownership, while $k_B^v$ is the `view key' since it can be used to find outputs spendable by Bob.

As will become clear in Chapter \ref{chapter:transactions}, without $k^o$ Alice can't compute the output's key image, so she can never know for sure if Bob spends the output she sent him.\footnote{Imagine Alice produces two transactions, each containing the same one-time output address $K^o$ that Bob can spend. Since $K^o$ only depends on $r$ and $K_B^v$, there is no reason she can't do it. Bob can only spend one of those outputs because each one-time address only has one key image, so if he isn't careful Alice might trick him. She could make transaction 1 with a lot of money for Bob, and later transaction 2 with a small amount for Bob. If he spends the money in 2, he can never spend the money in 1. In fact, no one could spend the money in 1, effectively `burning' it. Monero wallets have been designed to ignore the smaller amount in this scenario.}

Bob may give a third party his view key. Such a third party could be a trusted custodian, an auditor, a tax authority, etc. Somebody who could be allowed read access to the user’s transaction history, without any further rights. This third party would also be able to decrypt the output's amount (to be explained in Section \ref{sec:pedersen_monero}). See Chapter \ref{chapter:tx-knowledge-proofs} for other ways Bob could provide information about his transaction history.


\subsection{Multi-output transactions}
\label{sec:multi_out_transactions}

Most transactions will contain more than one output. If nothing else, to transfer `change’ back to the sender.\footnote{Actually, as of protocol v12 two outputs are {\em required} from each (non-miner) transaction, even if it means one output has 0 amount ({\tt HF\_VERSION\_MIN\_2\_OUTPUTS}). This improves transaction indistinguishability by mixing 1-output cases with the much more common 2-output transactions. Previous to v12 the core wallet software was already creating 0-value outputs. The core implementation sends 0-amount outputs to a random address.\marginnote{src/wallet/ wallet2.cpp {\tt transfer\_ selected\_ rct()}}}\footnote{After Bulletproofs were implemented in protocol v8, each transaction became limited to no more than 16 outputs ({\tt BULLETPROOF\_MAX\_OUTPUTS}). Previously there was no limit, aside from a restraint on transaction size (in bytes).}%BULLETPROOF_MAX_OUTPUTS; wallet2::transfer_selected_rct() for random address 0-amount

Monero senders usually generate only one random value $r$. The curve point $r G$ is typically known as the {\em transaction public key} and is published alongside other transaction data in the blockchain.\\

To\marginnote{src/crypto- note\_core/ cryptonote\_ tx\_utils.cpp {\tt construct\_ tx\_with\_ tx\_key()}}[-.8cm] ensure that all one-time addresses in a transaction with $p$ outputs are different even in cases where the same addressee is used twice, Monero uses an output index. Every output from a transaction has an index $t \in \{0, ..., p-1\}$. By appending this value to the shared secret before hashing it, one can ensure the resulting one-time addresses are unique:\marginnote{src/crypto/ crypto.cpp {\tt derive\_pu- blic\_key()}}[1.2cm]\vspace{.175cm}%construct_tx_with_tx_key() indexes 0 to p-1
\begin{align*}
  K_t^o &= \mathcal{H}_n(r K_t^v, t)G + K_t^s = (\mathcal{H}_n(r K_t^v, t) + k_t^s)G  \\ 
  k_t^o &= \mathcal{H}_n(r K_t^v, t) + k_t^s
\end{align*} 



\section{Subaddresses}
\label{sec:subaddresses}

Monero users can generate subaddresses from each address \cite{MRL-0006-subaddresses}. Funds sent to a subaddress can be viewed and spent using its main address’s view and spend keys. By analogy: an online bank account may have multiple balances corresponding to credit cards and deposits, yet they are all accessible and spendable from the same point of view – the account holder.\\

Subaddresses are convenient for receiving funds to the same place when a user doesn't want to link his activities together by publishing/using the same address. As we will see, most observers would have to solve the DLP to determine a given subaddress is derived from any particular address \cite{MRL-0006-subaddresses}.\footnote{Prior to subaddresses (added in the software update corresponding with protocol v7 \cite{subaddress-pull-request}), Monero users could simply generate many normal addresses. To view each address's balance, you needed to do a separate scan of the blockchain record. This was very inefficient. With subaddresses, users maintain a look-up table of (hashed) spend keys, so one scan of the blockchain takes the same amount of time for 1 subaddress, or 10,000 subaddresses.}

They are also useful for differentiating between received outputs. For example, if Alice wants to buy an apple from Bob on a Tuesday, Bob could write a receipt describing the purchase and make a subaddress for that receipt, then ask Alice to use that subaddress when she sends him the money. This way Bob can associate the money he receives with the apple he sold. We explore another way to distinguish between received outputs in the next section.%\footnote{Before subaddresses were implemented in April 2018, the Monero developers supported a method called payment IDs, which allowed receivers to differentiate between owned outputs. Payment IDs were text strings included in a transaction's `extra', miscellaneous, data. Recipients could ask senders to include the payment IDs when writing a transaction. Payment IDs could also be encoded in conjunction with so-called integrated addresses. Payment IDs and integrated addresses do not affect transaction verification, so they can still be implemented by anyone, but since they are currently deprecated in the most popular wallets we have chosen not to explain them here.}

%\footnote{Note that the index $i$ could just as easily be some password-generated hash $\mathcal{H}_n(x)$. This would allow an address owner to view his subaddress funds from his main address view key, but only be able to spend those funds by inputting a password. There are currently no wallet implementations password protecting subaddresses, nor any known plans to develop a wallet with that feature.}
Bob\marginnote{src/device/ device\_de- fault.cpp {\tt get\_sub- address\_ secret\_ key()}} generates his $i^\nth$ subaddress $(i = 1, 2, ...)$ from his address as a pair of public keys $(K^{v,i}, K^{s,i})$:\footnote{It turns out the subaddress hash is domain separated, so it's really $\mathcal{H}_n(T_{sub},k^v,i)$ where $T_{sub} = $``SubAddr". We omit $T_{sub}$ throughout this document for brevity.}\vspace{.175cm}
\begin{align*}
    K^{s,i} &= K^s + \mathcal{H}_n(k^v, i) G\\
    K^{v,i} &= k^v K^{s,i}
\end{align*}
\quad So,
\begin{alignat*}{2}
    K^{v,i} &= k^v&&(k^s + \mathcal{H}_n(k^v, i))G\\
    K^{s,i} &= &&(k^s + \mathcal{H}_n(k^v, i))G
\end{alignat*}
    

\subsection{Sending to a subaddress}
    
Let's say Alice is going to send Bob `0' amount again, this time via his subaddress $(K_B^{v,1}, K_B^{s,1})$.
\begin{enumerate}
	\item Alice generates a random number $r \in_R \mathbb{Z}_l$, and calculates the one-time address\vspace{.175cm}
	\[ K^o  = \mathcal{H}_n(r K_B^{v,1},0)G + K_B^{s,1} \]

	\item Alice sets $K^o$ as the addressee of the payment, adds the output amount `0' and the value $r K_B^{s,1}$ to the transaction data, and submits it to the network.
	
	\item Bob\marginnote{src/crypto/ crypto.cpp {\tt derive\_ subaddress\_ public\_ key()}} receives the data and sees the values $r K_B^{s,1}$ and $K^o$. He can calculate $k_B^v r K_B^{s,1} = r K_B^{v,1}$. He can then calculate $K'^{s}_B = K^o - \mathcal{H}_n(r K_B^{v,1},0)G$. When he sees that $K'^{s}_B = K^{s,1}_B$, he knows the transaction is addressed to him.\footnote{An advanced attacker may be able to link subaddresses \cite{janus-attack} (a.k.a. the Janus attack). With subaddresses (one of which can be a normal address) $K_B^1$ $\&$ $K_B^2$ the attacker thinks may be related, he makes a transaction output with $K^o = \mathcal{H}_n(r K_B^{v,2},0)G + K_B^{s,1}$ and includes transaction public key $r K_B^{s,2}$. Bob calculates $r K_B^{v,2}$ to find $K'^{s,1}_B$ but has no way of knowing it was his {\em other} (sub)address's key used! If he tells the attacker that he received funds to $K_B^1$, the attacker will know $K_B^2$ is a related subaddress (or normal address). Since subaddresses are outside the protocol's scope, mitigations are up to wallet implementers. No known wallets have done so, and any mitigation would only work for compliant wallets. Potential mitigations include: not informing attackers of received funds, including encrypted transaction private key $r$ in transaction data, a Schnorr signature on the shared secret using $K^{s,1}$ as the base point instead of $G$, and including $rG$ in transaction data and verifying the shared secret with $rK^{s,1} \stackrel{?}{=} (k^s + \mathcal{H}_n(k^v, 1))*rG$ (requires the private spend key). Outputs received by a normal address should also be verified. See \cite{janus-mitigation-issue-62} for a discussion of the last mitigation listed.}
	
	Bob only needs his private view key $k_B^v$ and subaddress public spend key $K^{s,1}_B$ to find transaction outputs sent to his subaddress.
	
	\item The one-time keys for the output are:\vspace{.175cm}
	\begin{align*}
		K^o &= \mathcal{H}_n(r K_B^{v,1},0)G + K_B^{s,1} = (\mathcal{H}_n(r K_B^{v,1},0) + k_B^{s,1})G  \\ 
		k^o &= \mathcal{H}_n(r K_B^{v,1},0) + k_B^{s,1}
	\end{align*}
\end{enumerate}

Now,\marginnote{src/crypto- note\_core/ cryptonote\_ tx\_utils.cpp {\tt construct\_ tx\_with\_ tx\_key()}} Alice's transaction public key is particular to Bob ($r K_B^{s,1}$ instead of $r G$). If she creates a $p$-output transaction with at least one output intended for a subaddress, Alice needs to make a unique transaction public key for each output $t \in \{0,...,p-1\}$. In other words, if Alice is sending to Bob's subaddress $(K_B^{v,1}, K_B^{s,1})$ and Carol's address $(K_C^v, K_C^s)$, she will put two transaction public keys \{$r_1 K_B^{s,1},r_2 G$\} in the transaction data.\footnote{In\marginnote{src/crypto- note\_basic/ cryptonote\_ basic\_impl.cpp {\tt get\_account\_ address\_as\_ str()}} Monero subaddresses are prefixed with an ‘8’, separating them from addresses, which are prefixed with ‘4’. This helps senders choose the correct procedure when constructing transactions.}\footnote{There is some intricacy to when additional transaction public keys are used (see the code path {\tt transfer\_selected\_rct()} $\rightarrow$ {\tt construct\_tx\_and\_get\_tx\_key()} $\rightarrow$ {\tt construct\_tx\_with\_tx\_key()} $\rightarrow$ {\tt generate\_output\_ephemeral\_keys()} and {\tt classify\_addresses()}) surrounding change outputs where the transaction author knows the recipient's view key (since it's himself; also the case for dummy change outputs, which are created when a 0-amount output is necessary, since authors generate those addresses). Whenever there are at least two non-change outputs, and at least one of their recipients is a subaddress, proceed in the normal way explained above (a current bug in the core implementation adds an extra transaction public key to transaction data even beyond the additional keys, which is not used for anything). If either just the change output is to a subaddress, or there is just one non-change output and it's to a subaddress, then only one transaction public key is created. In the former case, the transaction public key is $rG$ and the change's one-time key is (subaddress index 1, using $c$ subscript to denote the change's keys) $K^o = \mathcal{H}_n(k^v_c r G,t)G + K_c^{s,1}$ using the normal view key and subaddress's spend key. In the latter case the transaction public key $r K^{v,1}$ is based off the subaddress's view key, and the change's one time key is $K^o = \mathcal{H}_n(k^v_c*r K^{v,1},t)G + K_c^s$. These details help mingle a portion of subaddress transactions amongst the more common normal address transactions, and 2-output transactions which compose around 95\% of transaction volume as of this writing.}



\section{Integrated addresses}
\label{sec:integrated-addresses}

To differentiate between the outputs they receive, a recipient can request senders include a {\em payment ID} in transaction data.\footnote{Currently, Monero implementations only support one payment ID per transaction regardless of how many outputs it has.} For example, if Alice wants to buy an apple from Bob on a Tuesday, Bob could write a receipt describing the purchase and ask Alice to include the receipt's ID number when she sends him the money. This way Bob can associate the money he receives with the apple he sold.

At one point senders could communicate payment IDs in clear text, but manually including the IDs in transactions is inconvenient, and cleartext is a privacy hazard for recipients, who might inadvertently expose their activities.\footnote{These long-form (256 bit) cleartext payment IDs were deprecated in v0.15 of the core software implementation (coincident with protocol v12 in November 2019). While other wallets may still support them and allow their inclusion in transaction data, the core wallet will ignore them.} In\marginnote{src/crypto- note\_basic/ cryptonote\_ basic\_ impl.cpp {\tt get\_ account\_ integrated\_ address\_as\_ str()}} Monero, recipients can integrate payment IDs into their addresses, and provide those {\em integrated addresses}, containing ($K^v$, $K^s$, payment ID), to senders. Payment IDs can technically be integrated into any kind of address, including normal addresses, subaddresses, and multisignature addresses.\footnote{Within the authors' knowledge, integrated addresses have only ever been implemented for normal addresses.}

Senders addressing outputs to integrated addresses can encode payment IDs using the shared secret $r K_t^v$ and an XOR operation (recall Section \ref{sec:XOR_section}), which recipients can then decode with the appropriate transaction public key and another XOR operation \cite{integrated-addresses}. Encoding payment IDs in this way also allows senders to prove they made particular transaction outputs (e.g. for audits, refunds, etc.).\footnote{Since an observer can recognize the difference between transactions with and without payment IDs, using them is thought to make the Monero transaction history less uniform. Because of this, since protocol v10 the core implementation adds a dummy\marginnote{src/crypto- note\_core/ cryptonote\_ tx\_utils.cpp {\tt construct\_ tx\_with\_ tx\_key()}} encrypted payment ID to all 2-output transactions. Decoding one will reveal all 0s (this is not required by the protocol, just best practice).}


\subsubsection*{Encoding}

The\marginnote{src/device/ device\_de- fault.cpp {\tt encrypt\_ payment\_ id()}} sender encodes the payment ID for inclusion in transaction data\footnote{In Monero payment IDs for integrated addresses are conventionally 64 bits long.}\vspace{.175cm}
\begin{align*}
         k_{\textrm{mask}} &= \mathcal{H}_n(r K_t^v,\textrm{pid\_tag}) \\
      k_{\textrm{payment ID}} &= k_{\textrm{mask}} \rightarrow \textrm{reduced to bit length of payment ID}\\
  \textrm{encoded payment ID} &= \textrm{XOR}(k_{\textrm{payment ID}}, \textrm{payment ID})
\end{align*}

We include pid\_tag to ensure $k_{\textrm{mask}}$ is different from the component $\mathcal{H}_n(r K_t^v, t)$ in one-time output addresses.\footnote{In Monero, pid\_tag = {\tt ENCRYPTED\_PAYMENT\_ID\_TAIL} = 141. In, for example, multi-input transactions we compute $\mathcal{H}_n(r K_t^v, t) \pmod l$ to ensure we are using a scalar less than the EC subgroup order, but since $l$ is 253 bits and payment IDs are only 64 bits, taking the modulus for encoding payment IDs would be meaningless, so we don't.}


\subsubsection*{Decoding}

Whichever\marginnote{src/device/ device.hpp {\tt decrypt\_ payment\_ id()}} recipient the payment ID was created for can find it using his view key and the transaction public key $r G$:\footnote{Transaction data does not indicate which output a payment ID `belongs' to. Recipients have to identify their own payment IDs.}\vspace{.175cm}
\begin{align*}
         k_{\textrm{mask}} &= \mathcal{H}_n(k_t^v r G,\textrm{pid\_tag}) \\
      k_{\textrm{payment ID}} &= k_{\textrm{mask}} \rightarrow \textrm{reduced to bit length of payment ID}\\
          \textrm{payment ID} &= \textrm{XOR}(k_{\textrm{payment ID}}, \textrm{encoded payment ID})
\end{align*}

Similarly, senders can decode payment IDs they had previously encoded by recalculating the shared secret.%remove this line?



\section{Multisignature addresses}
\label{sec:multisignature-addresses}

Sometimes it is useful to share ownership of funds between different people/addresses. We dedicate Chapter \ref{chapter:multisignatures} to elaborating this topic.